\documentclass{article}

\usepackage[T1]{fontenc} 
\usepackage[french]{babel}
\usepackage[utf8]{inputenc}
\usepackage{graphicx}
\graphicspath{{./img/}}
\usepackage{amsmath, amssymb}
\usepackage[left=3cm,right=3cm,top=2cm,bottom=2cm]{geometry}
\usepackage{url}
\usepackage[linktocpage=true]{hyperref}

\title{Rapport de Projet : Tower Defense}
\author{Youri CHANCRIN, Théo DESCOMPS, Samuel KHALIFA, Jean-Baptiste MARTINEZ}
\date{\today}

\makeatletter
\let\mytitle\@title
\let\myauthor\@author
\makeatother

\begin{document}

\begin{flushleft}

\thispagestyle{empty}
\includegraphics[width=0.2\textwidth]{enseirb-matmeca}
\graphicspath{{./img/}}

\vspace{\stretch{0.4}}

\hrulefill \\[2em]
\begin{center}
\textbf{\Huge \mytitle}\\[0.8em]
{\large equipe : projetss6-tower-19048}
\\[0.8em]
\end{center}
\hrulefill

\vspace{\stretch{0.6}}

{\large \textbf{Auteurs :} \myauthor}\\
\vspace{\stretch{0.2}}
{\large \textbf{Responsable de projet :} David Renault}\\
\vspace{\stretch{0.2}}
{\large \textbf{Enseignant référent :} Frédéric Herbreteau}

\vspace{\stretch{1}}

\begin{center}
  Première année, filière informatique\\
  Date : \today
\end{center}

\vspace{\stretch{1}}

\hfill Version actuelle du document : draft

\end{flushleft}

\newpage

\tableofcontents

\newpage

\section{Sujet}
L'objectif de ce projet est la réalisation d'un jeu de type tower défense.

Ce genre peut se résumer en une seule phrase :\\[8px]
\centerline{\emph{\label{description_jeu} Des attaquants envahissent une zone défendue par des défenseurs.}}\\

La programmation fonctionnelle est le paradigme mis en avant.
L'exécutable est réalisé en Typescript.

\subsection{Theme}
Nous avons choisi de créer un jeu de tower defense autour du thème
du Pastafarisme.
Selon cette religion, la température de la Terre est inversement proportionnelle 
au nombre de pirates qu’elle contient.
Il faut donc répandre cette idéologie en défendant le monstre spaghetti volant et 
en augmentant le nombre de pirate sur Terre.

\section{Conception}

Nous débutons la conception avec notre description du jeu \ref{description_jeu} 
pour en tirer des élements fondamentaux.

On remarque en premier lieu la présence d'entités, des attaquants, mais aussi des 
défenseurs. Bien qu'elles soient antagonistes, ces entités ont des similarités, notamment 
dans le traitement de leurs actions. 

On décide donc de les regrouper sous le même nom, les acteurs.


\subsection{Entité Acteur}


Les acteurs forment un ensemble d'entités aux comportements et états variés 
qui intéragissent entre elles.

Le type acteur est très générique et peut représenter beaucoup de choses.
Pour garder une consistence entre nos acteurs, on définit des archétypes.

\begin{itemize}
   \item Spaghetti Monster : Le gentil monstre spaghetti que les défenseurs doivent protéger.
   \item Good Guy : Les défenseurs.
   \item Ignorant : Un type d'attaquant.
   \item Ignorance Spreader : Un autre type d'attaquant.
   \item Spawner : Un point d'apparition pour les attaquants.
   \item Ground : Un bout de chemin sur lequel les attaquants se déplacent.
\end{itemize}

Les types Good Guy, Ignorant et Ignorance Spreader représentent 
les défenseurs et les attaquants, ils découlent de notre démarche.
Cependant, nos acteurs comportent aussi un type Spawner et Ground.
Ces types ne peuvent pas être attaqués ou attaquer, mais ils peuvent interagir avec le reste des entités, ce qui justifie qu'ils
soient aussi considérés comme des acteurs.

En effet, Spawner peut être vu comme un attaquant immortel, et Ground 
un défenseur qui contraint le déplacement des attaquants.

Pour définir ces interactions, chaque acteur a besoin de disposer d'un ensemble 
de comportements qui lui sont propre. 

Concernant les comportements des acteurs, on veut qu'ils puissent évoluer 
au cours de la partie, comme pour simuler qu'un acteur subit un effet incapacitant 
ou stimulant.

Pour que ces interactions prennent effet, nos entités ont 
égalements besoin d'un état interne.

Les états internes des entités comportent toujours :

\begin{itemize}
    \item Une \textbf{position}
    \item Un \textbf{type} d'entité
\end{itemize}

Avec la diversification des acteurs vient le problème de la spécialisation qui se traduit par l'ajout d'attributs \emph{spécifiques} à chaque acteur.

Par exemple, les acteurs béligérants ont besoins d'attributs spécifiques tels que des points de croyance et une limite que ces points peuvent atteindre.

On pourrait rajouter tous les attributs nécessaires à tous les comportements 
à tous les acteurs. Mais c'est une mauvaise idée pour deux raisons.
Premièrement, cela entraînerait une modification sur le type acteur à chaque nouvel attribut ajouté, et entraînerai une modification de la signatures de certaines fonctions auxiliaires comme le constructeur.
Secondement, le type acteur aurait toujours plus d'états possibles, 
et deviendrait de plus en plus complexe et lourd à comprendre et gérer.

Pour éviter ces désagréments, nous avons séparé la responsabilité des attributs 
dépendants de certains comportements dans un champs \textbf{externalProps}.
Pour que ce champs puisse satisfaire toutes les possibilités d'attributs, il
est de type dictionnaire de quelconque. Toutefois, ce typage faible est compensé 
par la présence de fonctions getter et setter agissants comme une interface 
avec \textbf{externalProps}.

Nos acteurs sont donc parés à toutes éventalité d'évolution.
Cependant, bien qu'ils disposent de comportements, ils ne peuvent modifier leurs 
états internes directement.
Nous avons besoin de quelque chose d'ordre supérieur pour gérer leurs interactions.
Quelque chose comme un système de phase.

\subsection{Phase}

Nos acteurs effectuent des actions qui ont des conséquences.
Cela implique qu'il y a un avant et un après aux actions.
Pour chaque type d'action, on créer une phase.
Elle agit comme un ordonnanceur, elle obtient les intentions de chaque 
acteur pour l'action qu'elle traite, et applique les modifications, 
créant une nouvelle liste d'acteur.

Nommons un tour de jeu comme une execution de l'ensemble de nos phases.

Voici une liste des actions que les phases doivent permettre aux acteurs de réaliser :
\begin{itemize}
  \item Se déplacer
  \item Attaquer un autre acteur
  \item Soigner un autre acteur
  \item Faire apparaitre des nouveaux acteurs
  \item Se retirer du jeu
\end{itemize}  

Dans un soucis de cohérence et d'équité entre les acteurs, 
leurs intentions sont traitées de manière synchronisées au cours des phases.
Si ce n'était pas le cas, l'ordre dans lequel les acteurs jouent pourrait influer 
sur le résultat du tour de jeu. Par exemple, si un acteur joue dans sa 
phase d'attaque, inflige des dégats à un ennemi, puis joue sa phase de déplacements,
s'eloignant et privant l'ennemi d'une juste riposte.

Le découpage d'un tour de jeu en une liste de phase permet d'être modulaire.
Il en résulte qu'il est très simple de rajouter des phases de jeu, ou de 
changer l'ordre dans lequel les acteurs agissent.

Enfin, il peut arriver que les intentions des acteurs soient incompatibles.
Par exemple, si deux acteurs ayant des collisions veulent aller au même endroit.
Chaque phase doit pouvoir résoudre ses conflits.

Nous disposons donc d'acteurs, qui peuvent désormais intéragir entre eux avec 
une grande modularité.

Il ne nous manque plus qu'une seule notion issue de notre description du jeu 
\ref{description_jeu} , celle de zone.

On aimerait contraindre nos acteurs à rester dans celle-ci. Cela 
rejoint l'idée précédente d'intention d'un acteur à interdire, en l'occurence, 
on veut interdire les acteurs de quitter notre zone de jeu.

Mais cela n'est pas possible avec nos phases, car elles n'ont justement pas 
cette notion de zone.
On va donc remédier au problème en créant un objet supérieur, un objet représentant 
un monde.

\subsection{Monde}

Notre monde doit représenter une zone de jeu.
C'est un ensemble de coordonnées sur lesquelles nos entités peuvent exister.

Grâce à lui, on va réaliser deux opérations supérieur aux phases et aux acteurs que 
l'on ne pourrait faire autrement.
Tout d'abord, contraindre tous les acteurs à se trouver dans notre monde.
Ensuite, définir le chemin que les attaquants vont parcourir lors de leur 
invasion sur la zone de jeu.

Avec ces trois élements que sont les acteurs, les phases et le monde, on va pouvoir 
créer des acteurs de différents types, effectuant diverses actions, le tout dans une 
zone définie.

\section{réalisation}
\subsection{Phases, typage fort ou typage faible}
Choix : Typage fort
Parce que : Impose rigourosité et cohérence entre les types des propositions des 
acteurs et la fonction transformant les acteurs et permet d'expérimenter avec 
le riche système de type de Typescript, notamment les mapped types

Inconvénient : pas dynamique, mais pas un besoin pour nous donc pas important

- Actions par défaut.

Les petites sections comme c'est juste si le contenu vous inspire, sinon il faut supprimer
\subsubsection{Structure du monde et des acteurs}
Choix : Les acteurs sont séparés du monde
A cause de l'habitude de la programmation object, on aurait envie d'établir le plus de liens possible entre nos structures
Parce qu'on est dans la programmation fonctionnelle, on ne peut modifier l'attribut d'un 'object' dans le recréer, 
donc avoir les acteurs dans le monde est très impratique, car les acteurs sont sans cesse recréé, ceux qui forcerait de 
recréer le monde pour rien.

\subsubsection{Position des acteurs}
Chaque acteur contient sa position.
Si un acteur ne bouge pas, il n'est pas modifié.

Inconvénient : complexité n carré, n le nombre d'acteur, pour avoir tous les acteurs à une certaine distance d'un acteur.

Alternative : Faire un tableau 2D avec chaque acteur dans une case. Cependant, ce n'est pas une structure fonctionnelle, 
le projet portant sur cette dernière, on évite donc l'utilisation d'une matrice des positions.

\section{conclusion}
limite pureté?
- Dès qu'on fait du random, c'est plus pure
- Probleme de perf ?

\end{document}