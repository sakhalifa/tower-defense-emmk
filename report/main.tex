\documentclass{article}

\title{Rapport de Projet : Tower Defense}
\authors{Youri Chancrin, Théo Descomps, Samuel Khalifa, Jean-Baptiste Martinez}

\begin{document}

\maketitle

\section{Sujet}
Quel est le sujet
- attentes
 => Faire un jeu de type tower défense
 => Utiliser la programmation fonctionnelle
    - pureté, modularité, généralisation et spécialisation, fermetures, décorateurs, chaînage
 => contraintes :
    Éviter la programmation orientée objet.
    Type commun des entitées du jeu "Actor".
    Système de phases de jeu.
    Programmation en TypeScript, tests avec Jest, utilisation de ESLint, client web avec parcel

\section{Elements du jeu}
- Monde
- Acteurs
- Phases
- Boucle de jeu
- Importance des modules génériques (utils.ts, geometry.ts)

\section{Boucle de jeu}
- Un monde = plan 2d
- acteurs = séparé du monde.
Phases:
 - Acteur propose une action
 - Phase transforme les acteurs en fonction de leurs propositions.
 - Puis, le monde impose ses limites, corrigeant les acteurs invalides (en dehors du monde)


\section{Phases, typage fort ou typage faible}
Choix : Typage fort
Parce que : Impose rigourosité et cohérence entre les types des propositions des 
acteurs et la fonction transformant les acteurs et permet d'expérimenter avec 
le riche système de type de Typescript, notamment les mapped types

Inconvénient : pas dynamique, mais pas un besoin pour nous donc pas important

- Actions par défaut.

\section{Monde et acteur lié ?}
Choix : Les acteurs sont séparés du monde
A cause de l'habitude de la programmation object, on aurait envie d'établir le plus de liens possible entre nos structures
Parce qu'on est dans la programmation fonctionnelle, on ne peut modifier l'attribut d'un 'object' dans le recréer, 
donc avoir les acteurs dans le monde est très impratique, car les acteurs sont sans cesse recréé, ceux qui forcerait de 
recréer le monde pour rien.

\section{Position des acteurs}
Chaque acteur contient sa position.
Si un acteur ne bouge pas, il n'est pas modifié.

Inconvénient : complexité n carré, n le nombre d'acteur, pour avoir tous les acteurs à une certaine distance d'un acteur.

Alternative : Faire un tableau 2D avec chaque acteur dans une case. Cependant, ce n'est pas une structure fonctionnelle, 
le projet portant sur cette dernière, on évite donc l'utilisation d'une matrice des positions.

\section{Génération aléatoire des Acteurs}
- Génération aléatoire des chemins
- Génération aléatoire des Acteurs

\section{Détail de la structure Actor}
- externalProps

\section{Règles et fin de partie}

\section{limite pureté}
- Dès qu'on fait du random, c'est plus pure
- Probleme de perf ?

\end{document}